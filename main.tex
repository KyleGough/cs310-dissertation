\documentclass{article}
\usepackage[utf8]{inputenc}

\title{CS310 Dissertation - Using Swarm AI to map a Cave Network}
\author{Kyle Gough}
\date{October 2018}

\begin{document}

\maketitle

\tableofcontents

\section{Introduction}

Many unexplored caverns and mineshafts are dangerous for human exploration due to various factors such as flood risk, heat, tight spaces, harmful gases and instability. A safer and risk-free method would be to employ the use of a fleet of autonomous drones. The drones will use Swarm AI to avoid contact with each other and the cavern walls whilst simultaneously identifying potential unexplored paths and avoid crowding by splitting up wherever possible to increase the efficiency of the exploration. This project aims to identify and simulate the specific behaviours required to increase the efficiency of cave exploration.

\section{Background}

\section{Current Hardware and Software}

\section{Objectives}

\section{Possible Extensions}

\begin{itemize}
    \item Be able to generate visualisations and simulate solutions in 3D space.
    \item Use navigation techniques to find the best/safest routes to a specific point in the cave, or from any point in a cave to an exit.
\end{itemize}

\section{Methods}

\section{Testing}

\section{Timetable}

\section{Risk Assessment}

\section{Resources}

\begin{itemize}
    \item Git - For version control of the software and resources.
    \item Github - For storing the git repository remotely. Provides online version control and backups.
    \item Unity - For visualisation of the cavern environment and simulating the navigating drones.
    \item C#
\end{itemize}

\section{References}

\end{document}
